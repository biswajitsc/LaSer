%chemso --- Support for submissions to American Chemical
%  Society journals
% Maintained by Joseph Wright
% E-mail: joseph.wright@morningstar2.co.uk
% Originally developed by Mats Dahlgren
%  (c) 1996-98 by Mats Dahlgren
%  (c) 2007-2008 Joseph Wright
% Released under the LaTeX Project Public license v1.3c or later
% See http://www.latex-project.org/lppl.txt
% 
% Part of this bundle is derived from cite.sty, to which the
% following license applies:
%   Copyright (C) 1989-2003 by Donald Arseneau
%   These macros may be freely transmitted, reproduced, or
%   modified provided that this notice is left intact.
% ----------------------------------------------------------------
% 
% The achemso bundle provides a LaTeX class file and BibTeX style
% file in accordance with the requirements of the American
% Chemical Society.  The files can be used for any documents, but
% have been carefully designed and tested to be suitable for
% submission to ACS journals.
% 
% The bundle also includes the natmove package.  This package is
% loaded by achemso, and provides automatic moving of superscript
% citations after punctuation.

\documentclass[
%journal=ancac3, % for ACS Nano
%journal=acbcct, % for ACS Chem. Biol.
journal=jacsat, % for undefined journal
manuscript=article]{achemso}

\usepackage{titlesec}% http://ctan.org/pkg/titlesec
\titleformat{\section}%
  [hang]% <shape>
  {\normalfont\bfseries\Large}% <format>
  {}% <label>
  {0pt}% <sep>
  {}% <before code>
\renewcommand{\thesection}{}% Remove section references...
\renewcommand{\thesubsection}{\arabic{subsection}}%... from subsections

\usepackage[version=3]{mhchem} % Formula subscripts using \ce{}
\usepackage{hyperref}
\hypersetup{
    colorlinks=true,
    linkcolor=blue,
    filecolor=magenta,      
    urlcolor=cyan,
}

\newcommand*{\mycommand}[1]{\texttt{\emph{#1}}}

\author{Prithwish Mukherje}
\author{Biswajit Paria}
\author{Agnivo Saha}
\author{Haritabh Singh}
\author{Anurag Anand}
\author{Sandesh C}
\author{Mayank Singh}
\email{mayank4490@gmail.com}
\affiliation[IIT Kharagpur]
{Department of Computer Science, IIT, Kharagpur}


\title[\texttt{achemso} demonstration]
{LaSer : A search engine for mathematical formulae}

\begin{document}

\begin{abstract}
We present a search engine for mathematical formulae. The
MathWebSearch system harvests the web for content representations
(currently MathML) of formulae and indexes them using unigrams, 
bigrams and trigrams. The query for now is limited to Latex format only and can and will be extended to free form queries. Our Project can be found at \url{https://github.com/biswajitsc/LaSer}

\end{abstract}


\section{Introduction}

As the world of information technology grows, being able to quickly
search data of interest becomes one of the most important tasks in
any kind of environment, be it academic or not. This paper addresses
the problem of searching mathematical formulae from a semantic
point of view, i.e. to search for mathematical formulae not via their
presentation but their structure and meaning.


\section{Related Work}

We have seen that early work has been done in this fields, allowing users to search for mathematical equations. 
There are multiple sites available such as \url{http://latexsearch.com/}, \url{http://uniquation.com/en/}, \url{http://www.searchonmath.com/} 
that provide equation searching features. Most of the work done involves converting the equations available into xmls,
sustitution trees and indexing via various open source software. On other hand we are extracting various features other than simple 
xmls of the equations that help in increasing the efficiency of the system, as described in the following sections.

\section{Outline of workflow}

	\subsection{Convert Latex Formula to xml}
	
	The conversion from Latex formulae to xml has been performed using MathML\\
	e.g : $\lambda = 1$ is coverted to the following xml\\ 
	$<$?xml version="1.0" encoding="UTF-8"?$>$\\
	$<$m:math xmlns:m="http://www.w3.org/1998/Math/MathML" display="block"$>$\\
	  $<$m:mrow$>$\\
	    $<$m:mi$>$ $\lambda$ $<$/m:mi$>$\\
	    $<$m:mo$>$=$<$/m:mo$>$\\
	    $<$m:mn$>$1$<$/m:mn$>$\\
	  $<$/m:mrow$>$\\
	$<$/m:math$>$
	
	\subsection{Simplifying using equations and generating additional xmls.}
	
	The simpiflication has been done using python library sympy. The MathML's are converted into expressions and
	sympy's simplify routine is called and the corresponding MathML is generated. We work on both the original MathML
	and the simplified MathML.
	e.g : ((a + b) + c) is simplified to a + b + c and then MathML is generated as above.
	If the operands are constants, then the equation is simplified by evaluating it. e.g : 2 + 3.5 is replaced by 5.5 .
	
	\subsection{Number Normalization and Unicode Normalization.}
	
	  \subsubsection{Number Normalization}
	   Firstly, we are taking the MathML and taking the decimal numbers obtained inside $<mn> </mn>$ tags. We are
	   adding all the numbers obtained by reducing precision. For eg. - Original MathML :- \\
	   $<math xmlns="http://www.w3.org/1998/Math/MathML" display="block">$\\
	   $<mrow>$\\
	   $<mi>x</mi>$\\
	   $<mo> + </mo>$\\
	   $<mn>2.90646</mn>$\\
	   $</mrow>$\\
	   $</math>$\\
	   
	   Converted MathML :-\\
	   $<math xmlns="http://www.w3.org/1998/Math/MathML" display="block">$\\
	   $<mrow>$\\
	   $<mi>x</mi>$\\
	   $<mo> + </mo>$\\
	   $<mn>2.90646</mn> <mn>3.0</mn> <mn>2.9</mn> <mn>2.91</mn> <mn>2.906</mn> <mn>2.9065</mn>$\\
	   $</mrow>$\\
	   $</math>$\\

	  \subsubsection{Unicode Normalization}
	  Here we are normalizing the unicode characters to increase the match. We use NFKD: Normalization Form Compatibility Decomposition of the unicodes. Each unicode is converted into its NFKD form using python library unicodedata. So, if the base form of two unicode characters are same, they can still be matched.
	
	%\subsection{Operator Grouping }
	
	%Add description
	
	\subsection{Unigram + Bigram + Trigram Feature Extraction}
	
	Here the xml is treated simply as string and the unigram, bigram and trigram features are extracted. The corresponding
	postings list (inverted indexing of the equations) are also computed.\\
	Examples (The format is \{feature : [(eqn\_id, term\_frequency)]\} :- \\
	Unigram Features :- \{$<mi>ZN</mi> : [(2586, 1)]}, {<mn>&#x2147;</mn> : [(1338, 1), (4122, 1)]$\}\\
	Bigram Features :- \{$('<m:mo>\xe2\x88\xa7</m:mo>', '<m:mi>\\u03b2</m:mi>') : [(1807, 1), (1808, 3)]$\}\\
	Trigram Features :- \{$('<m:mn>56</m:mn>', '<m:mo>)</m:mo>', '<m:mi>f</m:mi>') : [(2143, 1)]$\}\\
	
	\subsection{TF-IDF weights calculation}
	
	Each equation is converted into a vector representation using the extracted unigram, bigram and trigram features\\
	The $tf_{weight}(eqn, feature) = 1 + log_{10}tf$, $idf_{weight}(feature) = 1 + log_{10}(N/df)$, here $tf_{weight}(eqn, feature)$ stands
	for the term frequency or the number of times the ``feature'' occurs in ``eqn'' and $idf_{weight}(feature)$ stands
	for how rarely/frquently the term occurs in the equations.
	
	\subsection{Query Normalization}
	
	Latex query given as input is also normalised as mentioned above to reduce unicode variants. The resulting equation is also simplified. Finally the query is converted into a high dimensional vector space using the extracted features mentioned above. 
	
	\subsection{Cosine Similarity}
	
	Currently cosine similarity is being used as the metric to determine the similarity between query vector and the indexed vectors. 
	
	\subsection{Backend and frontend}
	
	The Backend is running on a python server and webpage is used as the frontend



%\section{The LaSer Application}

%Add snapshots of UI here

\section {Conclusions and Future Work}
\begin{itemize}
	\item Support free form queries as input.
	\item Retrieve on basis of structural similarity, for example: $x + y = 1$ and $a + b$ and $(2*X - 1) + (2*Y)$ are all structurally similar. 
	\item Retrieve on simplification for example $x*x = 1$ and $x^2 = 1$ are same after simplification.
	\item Support search by names of popular equations.	

\end{itemize}

\section {References}
\begin{enumerate}
\item A Search Engine for Mathematical Formulae, Michael Kohlhase and Ioan A. Sucan
\item $5e^{x+y}$ : 	A Math Aware Search Engine(for CDS), Master Thesis Project, Arthur Oviedo, Supervisors: CERN: Nikolaos Kasioumis, EPFL: Karl Aberer

\end{enumerate}

\end{document}

